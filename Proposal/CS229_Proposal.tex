\documentclass[journal]{IEEEtran}
%
% If IEEEtran.cls has not been installed into the LaTeX system files,
% manually specify the path to it like:
% \documentclass[journal]{../sty/IEEEtran}





% *** CITATION PACKAGES ***
%
\usepackage{cite}
% cite.sty was written by Donald Arseneau
% V1.6 and later of IEEEtran pre-defines the format of the cite.sty package
% \cite{} output to follow that of the IEEE. Loading the cite package will
% result in citation numbers being automatically sorted and properly
% "compressed/ranged". e.g., [1], [9], [2], [7], [5], [6] without using
% cite.sty will become [1], [2], [5]--[7], [9] using cite.sty. cite.sty's
% \cite will automatically add leading space, if needed. Use cite.sty's
% noadjust option (cite.sty V3.8 and later) if you want to turn this off
% such as if a citation ever needs to be enclosed in parenthesis.
% cite.sty is already installed on most LaTeX systems. Be sure and use
% version 5.0 (2009-03-20) and later if using hyperref.sty.
% The latest version can be obtained at:
% http://www.ctan.org/pkg/cite
% The documentation is contained in the cite.sty file itself.



\usepackage{array,tabularx,hyperref}
\usepackage[margin=1in]{geometry}


% *** GRAPHICS RELATED PACKAGES ***
%
\ifCLASSINFOpdf
  \usepackage{graphicx}
  % declare the path(s) where your graphic files are
  \graphicspath{ {Images/} }
  \usepackage[justification=centering]{caption}
  % and their extensions so you won't have to specify these with
  % every instance of \includegraphics
  % \DeclareGraphicsExtensions{.pdf,.jpeg,.png}
\else
  % or other class option (dvipsone, dvipdf, if not using dvips). graphicx
  % will default to the driver specified in the system graphics.cfg if no
  % driver is specified.
  % \usepackage[dvips]{graphicx}
  % declare the path(s) where your graphic files are
  % \graphicspath{{../eps/}}
  % and their extensions so you won't have to specify these with
  % every instance of \includegraphics
  % \DeclareGraphicsExtensions{.eps}
\fi






% *** MATH PACKAGES ***
%
\usepackage{amsmath,float}


\begin{document}
\title{Campus Location Recognition using Audio and Visual Signals}

\author{James~Sun, (Teammate TBD)
        \\
        SUNetID:{Jsun2015}\\
        Email: \href{mailto:}{jsun2015@stanford.edu} }


% make the title area
\maketitle




\section{Introduction}
This project aims to create a system that can recognize (classify) geographical locations on the Stanford University Campus using Audio and Video signal inputs. The system will leverage labeled training data to train a classifier. This project will try to avoid focusing on sophisticated Image Processing techniques such as RANSAC/SIFT/SURF feature recognition, and instead utilize large quantities of simple features. Also, this project will attempt to emphasize audio based recognition as most previous work in this topic has been in computer vision.

\section{Related Work}
A previous CS229 course project identified landmarks based on visual features \cite{Crudge:article_typical}. \cite{Chen} gives a classifier that can distinguish between multiple types of audio such as speech and nature. \cite{Chu} investigates the use of audio features to perform robotic scene recognition. \cite{Chu2} takes a empirical approach toward recognizing environmental sound.

\section{Data}
Data is expected to be collected using a smartphone that has built-in audio and visual recording capabilities and has GPS capabilities. Training data will be labeled either by GPS data or by hand. Visual data will be captured by camera in still frames. Capturing video data by use of a GoPro\texttrademark camera is also possible. 

\section{Methods}
The goal is to have the system recognize natural geographic aggregates rather than recognize individual fine-grain coordinates. For example, the system will label a data set as belonging to "The Quad" or "Bytes Cafe", similar to how a person would naturally describe an environment. I expect to use supervised learning algorithms to separate data points based on audio and visual features.
\subsection{Audio Features}
As audio is potentially the more interesting data type, I have come up with a few basic features to evaluate. These include the following:
\begin{itemize}
\item Frequency Spectrum Bandwidth
\item Frequency Spectrum Variance
\item Frequency Spectrum mean
\item Intensity variance
\item Mean Intensity
\end{itemize}

% references section

% can use a bibliography generated by BibTeX as a .bbl file
% BibTeX documentation can be easily obtained at:
% http://mirror.ctan.org/biblio/bibtex/contrib/doc/
% The IEEEtran BibTeX style support page is at:
% http://www.michaelshell.org/tex/ieeetran/bibtex/
%\bibliographystyle{IEEEtran}
% argument is your BibTeX string definitions and bibliography database(s)
%\bibliography{IEEEabrv,../bib/paper}
%
% <OR> manually copy in the resultant .bbl file
% set second argument of \begin to the number of references
% (used to reserve space for the reference number labels box)
\bibliographystyle{IEEEtran}
\bibliography{CS229_Proposal}
\raggedbottom



\end{document}
